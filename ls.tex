\documentclass{article}

\usepackage[american]{babel}
\usepackage[utf8]{inputenc}
\usepackage{mysymbols}
\usepackage{amsmath}
\usepackage{amssymb}

\title{On the Lercier-Sirvent algorithm}
\author{Luca De Feo}

\begin{document}

\maketitle

\paragraph{Introduction}
The Lercier-Sirvent (LS) algorithm \cite{lercier+sirvent08} takes as
input two elliptic curves over a finite field (defined by a
Weierstrass equation) and a prime $\ell$ and outputs, if it exists, an
explicit isogeny of degree $\ell$ between them. It is a generalisation
of the BMSS method \cite{bostan+morain+salvy+schost08} to any
characteristic.

The original paper presenting LS only treats the case $p\ge5$. It is
not difficult to extend the algorithm to the case $p=3$ (chapter 8 of
my PhD thesis \cite{df+thesis} partly does the job), but the case
$p=2$ is trickier.

I'm gathering in this note the research I've done so far to adapt LS
to the case $p=2$.

\paragraph{BMSS}
BMSS takes as input two curves given by models
\begin{equation}
  \label{eq:1}
  \begin{gathered}
    E \;:\; y^2 = x^3 + ax + b \eqdef f(x),\\
    \tilde{E} \;:\; y^2 = x^3 + \tilde{a}x + \tilde{b} \eqdef \tilde{f}(x),
  \end{gathered}
\end{equation}
and assumes that these models are $\ell$-normalized, i.e. that the
$\ell$-isogeny $\I:E\to\tilde{E}$ is of the form
\begin{equation}
  \label{eq:2}
  \I(x,y) = \left(\frac{g(x)}{h(x)}, y\left(\frac{g(x)}{h(x)}\right)'\right)
\end{equation}
(Elkies \cite{elkies98} gives formulas to obtain normalized models
from the $\ell$-th modular polynomial). Plugging this equation into
the Weierstrass equations gives the differential equation
\begin{equation}
  \label{eq:3}
  f(x){\left(\frac{g}{h}\right)'}^2 = \tilde{f}\left(\frac{g}{h}\right).
\end{equation}
Set
\begin{equation}
  \label{eq:4}
  S(x) = \sqrt{\frac{h(1/\sqrt{x})}{g(1/\sqrt{x})}}, \qquad \frac{1}{\sqrt{S(1/\sqrt{x})}} = \frac{g(x)}{h(x)},
\end{equation}
then $S$ satisfies the equation
\begin{equation}
  \label{eq:5}
  \begin{gathered}
    G(x){S'}^2 = H(S), \quad S(0) = 1,\quad\text{with}\\
    G(x) = x^6f(1/x^2), \quad H(x) = x^6\tilde{f}(1/x^2).
  \end{gathered}
\end{equation}
Then, computing the isogeny boils down to find a power series solution
to this equation, invert the change of variables and reconstruct the
rational fraction $\frac{g}{h}$. It is easy to realize that at least
$4\ell-1$ coefficients of $S$ are needed in order to correctly compute
the isogeny.

\paragraph{Lercier-Sirvent}
Solving the differential equation \eqref{eq:5} implies taking
integrals, hence divisions by zero occur when $p<4\ell$. The idea of
LS is to lift the coefficients in the $p$-adics in order to allow
division by $p$: first the $j$-invariants of $E$ and $\tilde{E}$ are
lifted in $\Q_q$, then Elkies' formulas are applied to obtain
$\ell$-normalized models, finally BMSS is applied as above, yielding a
solution in $\Q_q$ that can be reduced modulo $p$. 

The key of the algorithm is that a precision of only $O(\log^2\ell)$
$p$-adic digits is required to correctly compute the reduced
isogeny. This bound is quite surprising and we will come back on it
later.

To also embed the case $p=3$, one considers curves given by equations
\begin{equation}
\label{eq:6}
  \begin{gathered}
    E \;:\; y^2 = x^3 + a_2x^2 + a_4x + a_6 \eqdef f(x),\\
    \tilde{E} \;:\; y^2 = x^3 + \tilde{a_2}x^2 + \tilde{a_4}x + \tilde{a_6} \eqdef \tilde{f}(x),
  \end{gathered}
\end{equation}
and all the previous equations are still verified, thus BMSS is
applied exactly the same way. It is easy to modify Elkies' formulas to
obtain $\ell$-normalized models of this form from the $\ell$-th
modular polynomial, and the $p$-adic precision needed is easily seen
to be the same as for $p\ge5$.

\paragraph{Even characteristic}
The case $p=2$ doesn't quite fit in this scheme. In this case, in fact, one must
consider curves given by equations
\begin{equation}
\label{eq:6}
  \begin{gathered}
    E \;:\; y^2 + xy = f(x),\\
    \tilde{E} \;:\; y^2 + xy = \tilde{f}(x).
  \end{gathered}
\end{equation}
By the change of variables $y\mapsto y-x/2$, it is immediate that $S$
satisfies the equation
\begin{equation}
  \label{eq:7}
  (G(x) + x^2/4){S'}^2 = H(S) + S^2/4,
\end{equation}
but a quick implementation of the method immediately shows that a
precision of $O(\log^2\ell)$ $p$-adic digits is no longer enough to
find a solution.

The reasons for the increased loss of $p$-adic precision are multiple.



\bibliographystyle{plain}
\bibliography{defeo}

\end{document}


% Local Variables:
% mode:TeX-PDF
% mode:reftex
% mode:flyspell
% ispell-local-dictionary:"american"
% End:

